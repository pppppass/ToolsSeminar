% !TeX encoding = UTF-8
% !TeX program = XeLaTeX
% !TeX spellcheck = en_US

% Author : 2prime
% Description : Outline: MXNet --- Seminar on Selected Tools Week 8

\documentclass[english, nochinese]{../TeXTemplate/pkuslide}

\usepackage{../TeXTemplate/def}

\title{Outline: PyTorch}
\subtitle{Seminar on Selected Tools Week 8 --- MXNet}

\author{Yiping Lu}
\date{Updated on May 12}

\subject{Outline: MXNet --- Seminar on Selected Tools Week 8}
\keywords{}

\begin{document}

\begin{frame}
\titlepage
\end{frame}

\begin{frame}
\tableofcontents[subsectionstyle=show]
\end{frame}

\section{Introduction}

\begin{frame}
\sectionpage
\end{frame}

\begin{frame}{Introduction}
Mxnet is a python package that provides two high-level features:
\begin{itemize}
\item Tensors On GPU!
\item A flexible and efficient library for deep learning.\textbf{both symbolic(based on computational graph like tensorflow) and command programming(like pytorch)}
\item High performance.
\item Opensource and full of meterials!
\end{itemize}
For more information, you can visit \url{http://mxnet.incubator.apache.org/}
\end{frame}

\section{Ndarray}

\begin{frame}
\sectionpage
\end{frame}

\begin{frame}{Ndarray}
\begin{enumerate}
\item mx.nd.empty((2, 3)) 
\item mx.nd.empty((2, 3), mx.gpu()) 
\item mx.nd.empty((2, 3), mx.gpu(2)) 
\item b = mx.nd.ones((2, 3))
\item a.copyto(b) (can do on different gpu/cpus)
\item b = a.asnumpy() 
\item a[:] = np.random.uniform(-0.1, 0.1, a.shape)  
\end{enumerate}

Understand \textbf{ctx=mx.gpu}
\end{frame}

\begin{frame}{Common Operation}
\begin{enumerate}
    \item  c = a.copyto(mx.gpu()) * b
    \item mx.nd.save('mydata.bin', [a, b])  
    \item c = mx.nd.load('mydata.bin') 
    
\end{enumerate}
\end{frame}


\section{Symbol}

\begin{frame}{Atuograd}

Invoking via \textbf{x.attach\_grad()}, and placing code in with autograd.record().

head\_gradient

MXNet.Symbol

\end{frame}

\begin{frame}{New Layer}

mx.operator.CustomOp : foward, backward

@mx.operator.register("dense")  

mx.operator.CustomOpProp

create\_operator, infer\_type, infer\_shape, list\_arguments, list\_output

fc1 = mx.symbol.Custom(data, name='fc1', op\_type='dense', num\_hidden=128)

\end{frame}

\section{Gluon}

\begin{frame}{Gluon}

Initialization via net = gluon.nn.Sequential() and adding layer in with net.name\_scope() by net.add()

net.collect\_params().initialize(mx.init.Xavier(magnitude=2.24), ctx=ctx)

softmax\_cross\_entropy = gluon.loss.SoftmaxCrossEntropyLoss()


trainer = gluon.Trainer(net.collect\_params(), 'sgd', {'learning\_rate': .1})

RK. Remember to with autograd.record()

\textbf{HybridBlock}
\end{frame}

\begin{frame}{Iteract With C/C++/Cuda Code}
\textbf{src/operator} is the dictionary of the definitions of operators in Symbol. You can add \textbf{-inl.h,.cc,.cu} three files and then complied the MXNet again.
\end{frame}

\section{Dataloader}
\begin{frame}{dataloader}
\textbf{mxnet.io.DataIter} provides a base class for you to define your own data loader. You need to define a \textbf{next} function and use the following function to return a batch \textbf{mxnet.io.DataBatch}

\end{frame}

\end{document}
